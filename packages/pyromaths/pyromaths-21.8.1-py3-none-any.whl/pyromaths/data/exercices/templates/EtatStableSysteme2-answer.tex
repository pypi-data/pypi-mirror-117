\exercice*
Puisque $P$ est l'état stable, alors $P=P\times M$.

\begin{align*}
    P\times M &= \begin{pmatrix}x&y\end{pmatrix}\times
\begin{pmatrix}
  (( a|facteur )) & (( (1-a)|facteur )) \\
  (( (1-b)|facteur )) & (( b|facteur )) \\
\end{pmatrix}\\
&= \begin{pmatrix}
  (( a|facteur("x") )) + (( (1-b)|facteur("y") )) & (( (1-a)|facteur("x") )) + (( b|facteur("y") ))
\end{pmatrix}
\end{align*}

Or $\begin{pmatrix}x&y\end{pmatrix}=P=P\times M$, donc les coefficients des matrices sont deux à deux égaux, donc $x=(( a|facteur("x") )) + (( (1-b)|facteur("y") ))$.

D'autre part, puisque $P$ est un état probabiliste, alors $x+y=1$, donc $y=1-x$. Donc, en remplaçant $y$ par $1-x$ dans l'équation précédente, on obtient :

\begin{align*}
  x &= (( a|facteur("x") )) + (( (1-b)|facteur ))\,(1-x) \\
  x &= (( a|facteur("x") )) + (( (1-b)|facteur )) - (( (1-b)|facteur("x") ))\\
  x-(( a|facteur("x") )) +(( (1-b)|facteur("x") )) &= (( (1-b)|facteur ))\\
  (1-(( a|facteur ))+(( (1-b)|facteur ))) \,x &= (( (1-b)|facteur ))\\
  (( (2-a-b)|facteur("x") )) &= (( (1-b)|facteur ))\\
  x &= \frac{(( (1-b)|facteur ))}{(( (2-a-b)|facteur ))}\\
x &= (( ((1-b)/(2-a-b))|facteur ))
\end{align*}

Enfin, puisque $y=1-x$, alors $y=1-(( ((1-b)/(2-a-b))|facteur ))=(( ((1-a)/(2-a-b))|facteur ))$.

L'unique état stable de ce graphe est donc $\begin{pmatrix}
(( ((1-b)/(2-a-b))|facteur )) &
(( ((1-a)/(2-a-b))|facteur ))
\end{pmatrix}$.
