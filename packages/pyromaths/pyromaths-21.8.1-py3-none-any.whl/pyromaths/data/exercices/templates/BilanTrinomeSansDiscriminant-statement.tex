\exercice
On considère le trinôme du second degré $f: x\mapsto (( a|facteur("X") )) (( b|facteur("*sox") )) (( c|facteur("so") ))$.

\begin{enumerate}
\item
\begin{enumerate}
    \item Montrer que pour tout $x\in\mathbb{R}$, on a : $f\,(x)=(( a|facteur )) \,\left( x (( -x1|facteur("so") )) \right) \, \left( x (( -x2|facteur("so") )) \right) $.
    \item Montrer que pour tout $x\in\mathbb{R}$, on a : $f\,(x)=(( a|facteur )) \,\left( x (( -alpha|facteur("so") )) \right)^2 (( beta|facteur("so") ))$.
\end{enumerate}
\item Résoudre les équations suivantes en choisissant la forme appropriée de $f$.
\begin{enumerate}
\item $f\,(x)=0$
\item $f\,(x)=(( c|facteur ))$
\item $f\,(x)=(( beta|facteur ))$
\end{enumerate}
\item
\begin{enumerate}
\item Dresser le tableau de variations de $f$.
\item Dresser le tableau de signes de $f$.
\end{enumerate}
\item Répondre aux questions suivantes en utilisant le tableau de signes ou de variations.
\begin{enumerate}
\item Résoudre $f(x)\geqslant0$.
\item Quel est l'extremum de $f$ ? Est-ce un maximum ou un minimum ? Pour quelle valeur de $x$ est-il atteint ?
\end{enumerate}
\end{enumerate}
