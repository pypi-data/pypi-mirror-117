\exercice
Dans un repère orthonormé, on cherche à déterminer l'équation d'une fonction dont la courbe passe par les points
$A\,( (( X[0]|facteur )) ~;~ (( Y[0]|facteur )) )$,
$B\,( (( X[1]|facteur )) ~;~ (( Y[1]|facteur )) )$ et
$C\,( (( X[2]|facteur )) ~;~ (( Y[2]|facteur )) )$.

On cherche un trinôme du second degré, c'est-à-dire une fonction $f$ définie sur $\interval[open]{-\infty}{+\infty}$ par
 \mbox{$f\,(x) = a\,x^2 + b\,x + c$} où $a$, $b$ et $c$ sont trois nombres réels, que l'on cherche à déterminer.

  \begin{enumerate}
    \item 
      \begin{enumerate}
        \item À partir des données de l'énoncé, écrire un système d'équations traduisant cette situation.
        \item En déduire que le système précédent est équivalent à : $M\,X = R$ avec

          $M = (( M|matrice ))$, $X= (( [["a"], ["b"], ["c"]]|matrice ))$ et $R$ une matrice colonne que l'on précisera.
      \end{enumerate}
  \end{enumerate}
  \begin{enumerate}
      \setcounter{enumi}{1}
    \item On admet que la matrice $M$ est inversible.
      Déterminer les valeurs des cœfficients $a$, $b$ et $c$, en détaillant les calculs.
    \item Quelle est la valeur de $f\,( (( x|facteur )) )$ ?

  \end{enumerate}
