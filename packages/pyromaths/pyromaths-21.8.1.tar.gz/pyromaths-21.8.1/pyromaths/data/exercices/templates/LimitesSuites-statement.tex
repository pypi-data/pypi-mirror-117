\exercice
Lorsqu'elles existent, calculer les limites des suites suivantes définies pour tout entier $n$ non nul.\par
\begin{tabularx}{\linewidth}[t]{XXX}
  (* for q in questions *)
    (* if q["cas"] == 1 *)
    $(( q["suite"] )) = (( q["polynome"] ))$
    (* elif q["cas"] == 2 *)
    $(( q["suite"] )) = (( q["polynome"] ))$
    (* elif q["cas"] == 3 *)
    $(( q["suite"] )) = \cfrac{(( q["polynome1"] ))}{(( q["polynome2"] ))}$
    (* elif q["cas"] == 4 *)
    $(( q["suite"] )) = \cfrac{(( q["polynome1"] ))}{(( q["polynome2"] ))}$
    (* elif q["cas"] == 5 *)
    $(( q["suite"] )) = \cfrac{(( q["polynome1"] ))}{(( q["polynome2"] ))}$
    (* endif *)
    (* if loop.index % 3 *)
    &
    (* else *)
    \\
    (* endif *)
  (* endfor *)
\end{tabularx}
