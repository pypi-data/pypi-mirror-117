\documentclass[10pt,reqno]{amsart}
\usepackage[left=.5in,right=.5in,top=.5in,bottom=.5in]{geometry}

%%\input{bhk-style}
\usepackage{pgf}
\usepackage[norelsize,ruled]{algorithm2e}   %%% without norelsize, arXiv chokes
\usepackage{booktabs,array,ragged2e}
\usepackage{pdflscape}
\usepackage{blkarray}% http://ctan.org/pkg/blkarray
\usepackage{etoolbox}

\newcommand{\matindex}[1]{\mbox{\scriptsize#1}}% Matrix index
\definecolor{mediumspringgreen}{rgb}{0.0, 0.98039215, 0.60392156}
\definecolor{darkgreen}{rgb}{0.0, 0.39215686274509803, 0.0}

\usepackage{hyperref}  

\newcommand{\R}{\mathbb R}
\newcommand{\Q}{\mathbb Q}
\newcommand{\Z}{\mathbb Z}
\newcommand{\N}{\mathbb N}
\renewcommand{\P}{\mathcal{P}}

\newcommand\CPL{\mathrm{CPL}}
\newcommand\DPL{\mathrm{DPL}}

\AtBeginDocument{ % no "subsubsubsection"
\let\subsectionautorefname\sectionautorefname
\let\subsubsectionautorefname\sectionautorefname
}

\chardef\Myunderscore=`\_
\newcommand\underscore{\Myunderscore\allowbreak}

\newcommand\githubsearchurl{https://github.com/mkoeppe/infinite-group-relaxation-code/search}
\input{../survey_graphics/sage-commands}
\pgfkeyssetvalue{/sagefunc/gj_2_slope}{\href{\githubsearchurl?q=\%22def+gj_2_slope(\%22}{\sage{gj\underscore{}2\underscore{}slope}}}%)
\pgfkeyssetvalue{/sagefunc/gj_forward_3_slope}{\href{\githubsearchurl?q=\%22def+gj_forward_3_slope(\%22}{\sage{gj\underscore{}forward\underscore{}3\underscore{}slope}}}%)
\pgfkeyssetvalue{/sagefunc/kzh_28_slope_1}{\href{\githubsearchurl?q=\%22def+kzh_28_slope_1(\%22}{\sage{kzh\underscore{}28\underscore{}slope\underscore{}1}}}%)
\pgfkeyssetvalue{/sagefunc/kzh_2q_example_1}{\href{\githubsearchurl?q=\%22def+kzh_2q_example_1(\%22}{\sage{kzh\underscore{}2q\underscore{}example\underscore{}1}}}%)
\pgfkeyssetvalue{/sagefunc/kzh_5_slope_fulldim_1}{\href{\githubsearchurl?q=\%22def+kzh_5_slope_fulldim_1(\%22}{\sage{kzh\underscore{}5\underscore{}slope\underscore{}fulldim\underscore{}1}}}%)
\pgfkeyssetvalue{/sagefunc/chen_4_slope}{\href{\githubsearchurl?q=\%22def+chen_4_slope(\%22}{\sage{chen\underscore{}4\underscore{}slope}}}%)
\pgfkeyssetvalue{/sagefunc/gmic}{\href{\githubsearchurl?q=\%22def+gmic(\%22}{\sage{gmic}}}%)
\pgfkeyssetvalue{/sagefunc/kzh_5_slope_fulldim_covers_1}{\href{\githubsearchurl?q=\%22def+kzh_5_slope_fulldim_covers_1(\%22}{\sage{kzh\underscore{}5\underscore{}slope\underscore{}fulldim\underscore{}covers\underscore{}1}}}%)
\pgfkeyssetvalue{/sagefunc/kzh_6_slope_fulldim_covers_1}{\href{\githubsearchurl?q=\%22def+kzh_6_slope_fulldim_covers_1(\%22}{\sage{kzh\underscore{}6\underscore{}slope\underscore{}fulldim\underscore{}covers\underscore{}1}}}%)
\pgfkeyssetvalue{/sagefunc/kzh_6_slope_1}{\href{\githubsearchurl?q=\%22def+kzh_6_slope_1(\%22}{\sage{kzh\underscore{}6\underscore{}slope\underscore{}1}}}%)
\pgfkeyssetvalue{/sagefunc/kzh_7_slope_4}{\href{\githubsearchurl?q=\%22def+kzh_7_slope_4(\%22}{\sage{kzh\underscore{}7\underscore{}slope\underscore{}4}}}%)
\pgfkeyssetvalue{/sagefunc/kzh_10_slope_1}{\href{\githubsearchurl?q=\%22def+kzh_10_slope_1(\%22}{\sage{kzh\underscore{}10\underscore{}slope\underscore{}1}}}%)
\pgfkeyssetvalue{/sagefunc/not_extreme_1}{\href{\githubsearchurl?q=\%22def+not_extreme_1(\%22}{\sage{not\underscore{}extreme\underscore{}1}}}%)
\pgfkeyssetvalue{/sagefunc/arithmetic_complexity}{\href{\githubsearchurl?q=\%22def+arithmetic_complexity(\%22}{\sage{arithmetic\underscore{}complexity}}}%)
\pgfkeyssetvalue{/sagefunc/number_of_slopes}{\href{\githubsearchurl?q=\%22def+number_of_slopes(\%22}{\sage{number\underscore{}of\underscore{}slopes}}}%)
\pgfkeyssetvalue{/sagefunc/number_of_components}{\href{\githubsearchurl?q=\%22def+number_of_components(\%22}{\sage{number\underscore{}of\underscore{}components}}}%)

\DeclareRobustCommand\sage[1]{\texttt{#1}}
\DeclareRobustCommand\sagefunc[1]{\pgfkeys{/sagefunc/#1}}

%%%%

\title[Infinite Group Problem Code: figures test suite]{Infinite Group Problem Code:\\figures test suite}

\thanks{The authors acknowledge partial support from the National Science
  Foundation through grant DMS-1320051 awarded to M.~K\"oppe.}

\begin{document}

\setcounter{page}{9}
\setcounter{section}{3}

\section{Figures from \itshape New computer-based search strategies for
  extreme functions\\ of the Gomory--Johnson infinite group problem}

\begin{figure}[h]
\centering
\includegraphics[width=.4\linewidth]{gj2s_restriction.pdf}
\quad
\includegraphics[width=.4\linewidth]{gj2s_interpolation.pdf}
\caption{The 2-slope extreme function
  \sagefunc{gj_2_slope}, discovered by Gomory and Johnson. \textit{Left},
  \sagefunc{gj_2_slope} for the finite group 
  problem with $q=5$ and $f=\frac{3}{5}$, obtained by \sage{\sagefunc{restrict_to_finite_group}(\sagefunc{gj_2_slope}())}. It is a discrete function whose interpolation is the right subfigure. \textit{Right}, \sagefunc{gj_2_slope} for the infinite group problem with $f=\frac{3}{5}$. It is a continuous piecewise linear function with two slopes, although it has four pieces. Its restriction to $\frac{1}{5}\Z$ is the left subfigure.}
\label{fig:gj2s_restriction_interpolation}
\end{figure}%

\begin{figure}[h]
\centering
\includegraphics[width=.6\linewidth]{q_v_grid.pdf}
\caption{The $q \times v$ grid discretization of the space of   continuous piecewise linear functions with rational data. Here $q=8$ and $v=6$.
%The continuous piecewise linear function $\pi$ has breakpoints in $\frac{1}{q}\Z$. 
%The function value at $\frac{i}{q}$ and the slope value on $[\frac{i-1}{q}, \frac{i}{q}]$ are denoted by $\pi_i$ and $q s_i$ respectively. The $\pi_i$'s and $s_i$'s can be discretized in $\{0, \frac{1}{v}, \dots, \frac{v-1}{v}, 1\}$.
}
\label{fig:q_v_grid}
\end{figure}

\begin{figure}
\centering
\includegraphics[width=\linewidth]{kzh_28_slope.pdf}
\caption{A 28-slope extreme function \sagefunc{kzh_28_slope_1} found by our search code. Each color in the plotting corresponds to a different slope value.}
\label{fig:kzh_28_slope}
\end{figure}

\begin{figure}
\centering
\includegraphics[width=\linewidth]{kzh_2q_move.pdf}
\caption{The example \sagefunc{kzh_2q_example_1}, showing that an oversampling factor of $m=3$ is best possible.}
\label{fig:kzh_2q_move}
\end{figure}

\addtocounter{figure}{4}

\begin{figure}[h]
\centering
\includegraphics[width=.42\linewidth]{kzh_5_slope_fulldim.pdf}
\quad
\includegraphics[width=.49\linewidth]{kzh_5_slope_fulldim-2d_diagram.pdf}
\caption{The 5-slope extreme function
  \sagefunc{kzh_5_slope_fulldim_1}
  found by our search code (\textit{left}). Its two-dimensional polyhedral
  complex $\Delta\P$ (\textit{right}), as plotted by the command
  \sage{\sagefunc{plot_2d_diagram}(h,colorful=True)}, does not have any
  lower-dimensional maximal additive faces except for the symmetry reflection
  or $x=0$ or $y=0$.} 
\label{fig:kzh_5_slope_fulldim}
\end{figure}

\addtocounter{figure}{1}

\begin{figure}[h]%
\includegraphics[width=.49\linewidth]{pattern_s6_q58.pdf}
\includegraphics[width=.49\linewidth]{pattern_s10_q166.pdf}
\caption{Special patterns on the two-dimensional polyhedral complex $\Delta\P_{\frac1q\Z}$. 
\textit{Left,} the $\Delta\P_{\frac1q\Z}$ of the $6$-slope extreme function
\sagefunc{kzh_6_slope_1} with $q=58$. We observe that the additive triangles
are located in the lower left and upper right corners. The function has the
same slopes on the intervals that are projections of the same color additive
triangles. The 6-pointed star patterns appear several times. \textit{Right,} the
lower-left corner of $\Delta\P_{\frac1q\Z}$ of the $10$-slope extreme function
\sagefunc{kzh_10_slope_1} with $q=166$, where we see that the 6-pointed stars
are actually the result of additivity patterns within certain intersecting
quadrilaterals (\emph{black}), which connect like
links of three chains.}
\label{fig:special_patterns}%
\end{figure}%
%

\end{document}

%%% Local Variables:
%%% mode: latex
%%% TeX-master: t
%%% End:
